\documentclass{article}

\usepackage{fullpage}
\usepackage{hhline}
\usepackage{hyperref}

\author{Rohit Saily, David Thompson, Raymond Tu}
\date{\today}
\newcommand{\mytitle}{4TB3 Project Development Plan}
\title{\mytitle}

\begin{document}

% Use own title
{\sffamily
{\huge \mytitle}

\medskip

David Thompson, 400117555~~~~~
Rohit Saily, 400144124~~~~~
Raymond Tu, 400137262~~~~~\today
}

\par\noindent\rule{\textwidth}{0.4pt}
\bigskip

\section{Project Description}

\subsection{Research Question}

How does the compilation time of the P0 Compiler differ between a sequential implementation and a concurrent implementation using Goroutines in Golang?  

\subsection{Developmental Challenges} % Rohit

\subsection{Test Plan} % David

\subsection{Documentation} % Raymond - Godoc

For the documentation of our project we will be using GoDoc. GoDoc is similar to documentation 
generators for other programming languages, for example JavaDoc for Java and Doxygen for Python. 
GoDoc generates documentation in PDF or HTML format from comments in the source code. 
Following the standard conventions for GoDoc will provide easily readable source code and good 
documentation for all methods implemented in the P0 Compiler. GoDoc is chosen because it is the standard documentation standard for Golang, 
which is the main programming language we will be using for this project. The conventions for a standard GoDoc comment to generate HTML
 is easy to read, and alone would already provide readable source code to any viewer. 
 In conclusion, GoDoc is intuitive to use and similar to existing documentation standards we have experience with, 
 which makes it the perfect choice for this project.

\section{Resources}

\subsection{Transcompilation} % David


Python to Go transpiler from Google: \url{https://github.com/google/grumpy}
Reference to the inception of the project: \url{https://groups.google.com/forum/#!topic/golang-nuts/tu75K5odt8U}

\subsection{Development} % Rohit

\subsection{Testing} % Raymond

\subsection{Poster Research} % David

\section{Division of Labour} % Everyone

\section{Weekly Schedule} % Everyone

\begin{center}
\begin{tabular}{ l | l }
  \textbf{Week} & \textbf{Deliverables} \\
  \hhline{=|=}
  Week 1 & Development Plan and Proposal\\
         & Setting Development Environment\\ 
  Week 2 & Transcompilation and Beginnning of Development\\
  Week 3 & Working Prototype\\
  Week 4 & Testing and Refinement of Prototype and Documentation\\
  Week 5 & Poster Presentation Creation\\
  Week 6 & Poster Presentation Creation\\
\end{tabular}
\end{center}

% Table

\end{document}
