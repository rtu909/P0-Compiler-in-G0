\documentclass{article}

\usepackage{fullpage}
\usepackage{hhline}
\usepackage{hyperref}

\author{Rohit Saily, David Thompson, Raymond Tu}
\date{\today}
\newcommand{\mytitle}{4TB3 Project Development Plan}
\title{\mytitle}

\begin{document}

% Use own title
{\sffamily
{\huge \mytitle}

\medskip

David Thompson, 400117555~~~~~
Rohit Saily, 400144124~~~~~
Raymond Tu, 400137262~~~~~\today
}

\par\noindent\rule{\textwidth}{0.4pt}
\bigskip

\section{Project Description}

\subsection{Research Question}

How does the compilation time of the P0 Compiler differ between a sequential implementation in Golang and a concurrent implementation in Golang that uses Goroutines?  

\subsection{Relevant Work}

Many compilers forego concurrent compilation of parts of one file for two
reasons.
First, the concurrency mechanisms provided by the languages that other compilers
are coded in have too much overhead.
The time it takes to set up tasks to compile the code concurrently exceeds the
time that could be saved by compiling concurrently.
Next, it is much easier to compile a group of files concurrently, with one
instance of the compiler for each file.
This keeps each core of the processor occupied, so additional concurrency on top
of this may not improve the compilation time by much.

However, Golang's compiler implements compiling different portions of one file
concurrently (see \url{https://golang.org/doc/go1.9#parallel-compile}).
It is a self-hosted compiler, i.e., it is written in Golang which is also the language it compiles to.
This means it can benefit from the cheap concurrency that the language provides
in order to facilitate the concurrent compilation.

\subsection{Developmental Challenges}
When developing this project, a few challenges need to be overcome to ensure the 
project is completed correctly and on time.

One major challenge is ensuring that the only significant difference between the P0 compiler
and its concurrent counterpart is that the first is a sequential program and the
latter is a concurrent program. This is necessary to ensure that any differences in compilation time can be primarily attributed to whether or not the compiler was implemented as a concurrent program. This challenge will be overcome by:
\begin{enumerate}
	\item Transcompiling the original P0 compiler from Python to Golang. This ensures that the
	compilers are compiled to machine code under the same principles.
	\item Executing tests for each implementation of P0 within the same runtime environment. This ensures that the machine code
	is executed in the same way for both implementations of P0.
	\item Testing the compilation time of both compilers for the same set of inputs. This prevents
	compilation time differences to be attributed to different inputs.
\end{enumerate}
\bigskip
Another major challenge is designing the concurrent implementation of the P0 compiler.\\
First, designing a concurrent program is difficult in general since it requires consideration of possible interleavings of program instructions. This aspect of the challenge will be overcome by referencing  material from \href{https://www.cas.mcmaster.ca/~se3bb4/}{McMaster University's SFWRENG 3BB4 course}.\\
Second, making a \textit{compiler} concurrent is challenging since various dependencies between statements need to be respected when separating the compilation into concurrent processes. To overcome this the program to be compiled will be decomposed into independent units which can be treated as their own programs and compiled in concurrent processes.

\subsection{Test Plan}

We will develop a test suite of example programs which encompass the features of
the P0 language. We will pull from the example programs in the notebooks, and
write our own.
We will also develop syntactically incorrect programs, or programs with type
errors, in order to test our error detection and reporting.

Using these programs, we will test each of the compilers' components independently as follows:

\begin{itemize}
  \item Scanner - Given the source code, does the scanner generate the expected
    sequence of tokens?
  \item Parser - Given the tokens corresponding to the source code, does the
    parser generate the expected AST?
  \item Generators - Given an AST of the source code, is the expected machine
    instructions generated?
\end{itemize}

We will also perform integration tests by running all the steps of the compiler
at once, ensuring that code generated by our compiler is identical to that of
the existing implementation.

In addition to the above, we will unit test the symbol table.
This particular program is more suited to unit testing than the other components,
because individual searches and additions can be made.

Finally, we will time the execution of the entire P0 Compilers to determine how much of a
time improvement concurrency provides.
In order to test the timing effectively, we will need to have a sizable program
written in P0.
We will use many programs of varying sizes to see how program size
affects the compile time.
We will also use a scripting language to time the compilation time of the transpiled
version of the original compiler and the concurrent version of the compiler.
The time comparisons between the compilers presented on the poster must be all
from the same computers.

\subsection{Documentation}

For the documentation of our project we will be using GoDoc. GoDoc is similar to documentation 
generators for other programming languages, for example JavaDoc for Java and Doxygen for Python. 
GoDoc generates documentation in PDF or HTML format based on comments in the source code. 
Following the standard conventions of GoDoc will make the source code readable and provide thorough 
enough documentation for each method that makes up the concurrent implementation of the P0 Compiler. GoDoc is chosen because it is the official documentation standard for Golang, 
which is the main programming language we will be using for this project. 
 In conclusion, GoDoc is intuitive to use and similar to existing documentation standards we have experience with, 
 which makes it the perfect choice for this project.
 
Information about GoDoc can be found at: \url{https://blog.golang.org/godoc-documenting-go-code}

\section{Resources}

\subsection{Transcompilation}

Python is generally an interpreted language, but Golang is compiled before execution.
This means that the Golang implementation of the P0 compiler would likely run
faster, even if no concurrency is used.
In order to get a fair comparison between the compilers, a few methods are
available.

The existing P0 compiler could be manually rewritten to Golang.
This method will take a long time, and would leave less time for refining the
concurrent Golang implementation.

Another method would be to compile the Python P0 compiler and the concurrent Golang
P0 compiler to a common language, then compare their run times in that 
language.
This is not ideal, because concurrency works differently between programming
languages.
Golang is known for having very efficient concurrency compared to other languages.
As a result, the efficiency of the concurrent Golang implementation may be
understated.

The preferred method is to transpile the existing Python code to Golang instead of
manually translating the program.
Google offers a Python to Golang transpiler called
\href{https://github.com/google/grumpy}{Grumpy}.


\subsection{Development}
To meet the deadline of the project, an IDE will be used to make the development of the concurrent
P0 compiler relatively easier. \href{https://www.jetbrains.com/go/}{GoLand} will be used as the IDE
since it is designed for Golang and offers a wide variety of developmental features that automate
many tedious programming tasks.\\
To develop the concurrent implementation of the P0 compiler, many concurrent design methods
need to be utilised carefully to avoid concurrency issues. Methods from \href{https://www.cas.mcmaster.ca/~se3bb4/}{McMaster University's SFWRENG 3BB4 course}. will
be applied during devlopment.\\

\subsection{Testing} % Raymond

For testing for errors in the P0 Compiler in Go, unit testing will be conducted. This unit testing will be done in the standard testing package that is
included in Go. This testing package also includes statement-based code coverage checking that will be useful as a metric for how well the code is tested.
The difference between this package and other standard unit testing for other languages such as PyTest for Python is that it does not utilize assert statements.
Instead an error is thrown after a boolean check coded by the programmer. This will allow the test package to detect multiple errors in the program, because
while an assert statement would stop the program after the first error, the test package in Go will instead detect the error and continue running other test cases.
In conclusion, using the Go test package we can eliminate errors in the sequential and concurrent implementation in Go before we test the differences in their runtimes
and answer the defined research question. \\    

For timing the execution time of the P0 Compiler, we can use a variety of different time libraries included in Python or Go. To be consistent, one timing library 
will be chosen
and used for testing the execution time of both implementations of the P0 Compiler. This will ensure 
accurate results to draw a conclusion from to answer the research question. How these libraries will be used for testing will be elaborated on in the Test Plan section.

\subsection{Poster} % David

Since Golang implements concurrency in its compiler, we can reference the
compilers code and documentation when talking about existing work.
As mentioned above, very few other compilers use concurrency.
More rigorous research will be needed in order to support this claim or
find examples of other concurrent compilers.

\section{Division of Labour} % Everyone

\begin{itemize}
  \item Development Plan and Proposal
  \begin{itemize}
    \item Research Question - Raymond
    \item Relevant Work - Davod
    \item Development Challenges - Rohit
    \item Test Plan - David
    \item Documentation - Raymond
    \item Transcompilation Resources - David
    \item Development Resources - Raymond
    \item Testing Resources - Raymond
    \item Poster Resource - David
  \end{itemize}
  \item Transcompilation - David
  \item Development - Everyone
  \begin{itemize}
    \item Scanner - Raymond
    \item Parser - Rohit
    \item WASM Generator - Raymond
    \item MIPS Generator - Rohit
    \item Symbol Table - David
  \end{itemize}
  \item Testing - Everyone
  \begin{itemize}
    \item Unit testing - David
    \item Writing P0 programs to test compilation - Raymond
    \item Timing testing script - Rohit
  \end{itemize}
  \item Poster - Everyone
  \begin{itemize}
    \item PDF - Everyone
    \item Demo - Everyone
    \item Script for presentation - Everyone
  \end{itemize}
\end{itemize}

\section{Weekly Schedule} % Everyone

\begin{center}
\begin{tabular}{ l | l }
  \textbf{Week} & \textbf{Deliverables} \\
  \hhline{=|=}
  Week 1 & Development Plan and Proposal\\
         & Setting Development Environment\\ 
  Week 2 & Transcompilation and Beginning of Development\\
  Week 3 & Working Prototype\\
  Week 4 & Testing and Refinement of Prototype and Documentation\\
  Week 5 & Testing and Poster Presentation Creation\\
  Week 6 & Poster Presentation Creation\\
\end{tabular}
\end{center}

% Table

\end{document}
